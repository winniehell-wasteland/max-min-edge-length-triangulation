\chapter{Triangulations}
One day Greg had art class and the children were given sheets of
paper with points on them. The task was to connect all the points 
such that they have as many bounded areas as possible. Greg 
connected all the points taking care that the connections would only 
intersect in the given points. He was happy to see that the result 
consisted only of triangles filling up the whole space within the 
points. Also he had more bounded areas than any other child --- 
even more than Joey who filled up his sheet with quadrangles. Greg's
teacher, Mrs. Minerva, told him that he made a triangulation.

\begin{definition}[Triangulation]\label{def:triangulation_subdivision}
As defined in \cite{deberg_compgeom} (section 9.1), the
triangulation \(T\) of a planar point set \(P\) is a maximal planar 
subdivision where \(P\) are the vertices.
\end{definition}

After seeing Greg's sheet, some of his friends let their connections
cross each other to get more areas, but Greg considered that
cheating. To stay the winner of the competition he made a new rule
that no point connections are allowed to cross.

\begin{definition}[Crossing]\label{def:crossing}
Two line segments \(s_i=(p_i,q_i)\) and \(s_j=(p_j,q_j)\) with 
different slope are \gls{cross}, if their intersection is not empty
and not an endpoint, i.e.

\[
  s_i, s_j~\gls{cross}
  \iff
  p = s_i\cap s_j,~
  p~\text{point},~
  p \not\in \{p_i,q_i,p_j,q_j\} \]

Two segments \(s_i\) and \(s_j\) are \gls{ncross} if they are
not \gls{cross}. A set \(S\) of segments is \gls{cross} if at least
two segments \(s_i, s_j \in S\) are \gls{cross}. It is \gls{ncross}
if each pair \(s_i, s_j \in S\) is \gls{ncross}.
\end{definition}

Joey tried to circumvent Greg's rule by adding new points to his
sheet. For each new point he added on the outside he could make
several new connections and therefore got soon many more areas. So
Greg had to quickly come up with another rule only allowing the
existing points to take part of the competition.

\begin{definition}[Induced Segments]\label{def:induced_segments}
The set of induced line segments \(S_P\) of a planar point set \(P\)
consists of all line segments which have endpoints in \(P\):

\[ S_P := \{\{p,q\} : p,q\in P, p\neq q\} \]

Hereby we consider the line segment from a point \(p\) to a point
\(q\) as equivalent to the line segment from \(q\) to \(p\) as 
replacing one by the other does not change the \gls{MMLT}:

\[ \{p, q\} \equiv \{q, p\} \]
\end{definition}

With those two new rules no one could think of a better way to make
areas than Greg's triangles. Even Mrs. Minerva said that under those
conditions it would not be possible to come up with a subdivision
which has more faces than a triangulation.

\begin{definition}[Triangulation]\label{def:triangulation}
The \cref{def:triangulation_subdivision} is equivalent to saying
a triangulation \(T\) of a planar point set \(P\) is a maximal
\gls{ncross} subset of the induced segments \(S_P\) of \(P\).
\end{definition}

After class was over, Greg asked Mrs. Minerva if there are different
kinds of triangulations. 
%--------------------------------------------------------------------##########

Minimum weight triangulation \cite{mwt_complexity}

 This is 


\todo[inline]{Write about related work}
\begin{itemize}
  \item Edelsbrunner paper -- open problem \cite{mmlt_problem}
  \item Sándor's paper --  NP-complete \cite{mmlt_complexity}
  \item Christiane's paper -- MMLT in polygons \cite{mmlt_polygons}
  \item Hu paper -- convex polygons in linear time \cite{mmlt_convex_polygons}
  \item Delaunay triangulation
\end{itemize}
