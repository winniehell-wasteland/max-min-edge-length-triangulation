\documentclass[
	a4paper,
	abstracton,
	bibtotoc,
	english,
  ngerman,
	twoside
] {scrreprt}

\usepackage[utf8]{inputenc}
\usepackage[T1]{fontenc}

\usepackage{babel}

\usepackage{
	amsmath,
	amssymb,
	cancel,
	csquotes,
	etoolbox,
	floatpag,
	framed,
	geometry,
	hyperref,
	isodate,
	microtype,
	multirow,
%	lscape,
	pdflscape,
	pdfpages,
%	stackengine,
	subcaption,
	todonotes,
	upgreek,
	url,
	xfrac,
	xcolor,
%	xspace,
}

\usepackage[algochapter,english,linesnumbered,ruled,vlined]{algorithm2e}
\renewcommand{\algorithmcfname}{Algorithm}

\usepackage[alldates=iso8601,backend=biber]{biblatex}
\usepackage[acronym,style=listdotted]{glossaries}
\usepackage[amsmath,framed,hyperref]{ntheorem}
\usepackage[section]{placeins}
 
% after hyperref
\usepackage{
	cleveref
}

% meta data
\newcommand{\docadvisor}{Prof.~Dr.~Sándor Fekete}
\newcommand{\docauthor}{Winfried Hellmann}
\renewcommand{\docdate}{2013-08-01}
\newcommand{\docsubject}{Masterarbeit}
\newcommand{\doctitle}{\foreignlanguage{english}{Computational Aspects of%
\texorpdfstring{\linebreak}{} MaxMin Triangulations}}

\author{\docauthor}
\date{\isodate{\docdate}}
%\date{\today~(DRAFT)}
\publishers{\small\textbf{Betreuer: }\docadvisor}
\subject{\docsubject}
\title{\doctitle}
%\subtitle{or: How Greg Found Large Triangles}
\titlehead{Institut für Betriebssysteme und Rechnerverbund\\
Technische Universität Braunschweig}

\hypersetup{
       pdfauthor = {\docauthor},
       pdftitle = {\doctitle ---},
       pdfsubject = {\docsubject},
       pdfkeywords = {maxmin, length, triangulation}
}

% underline links
\hypersetup{
  colorlinks=false,
  linkbordercolor=red,
  pdfborderstyle={/S/U/W 1}
}

\hypersetup{hidelinks}

%\bibliography{masterarbeit}
\addbibresource{masterarbeit.bib}
\nocite{*}

\theoremstyle{break}
\theorembodyfont{\normalfont}

\newcommand{\samepagetheorem}{%
\theoremprework{\begin{samepage}}%
\theorempostwork{\end{samepage}}%
}
%\theoremprework{\begin{minipage}{\textwidth}}
%\theorempostwork{\end{minipage}}

\samepagetheorem
\newframedtheorem{theorem}{Theorem}[chapter]

\samepagetheorem
\newframedtheorem{definition}[theorem]{Definition}

\samepagetheorem
\newframedtheorem{problem}[theorem]{Problem}

\theoremseparator{:}
\theoremstyle{nonumberbreak}
\samepagetheorem
\newtheorem{proof}{Proof}

\theoremstyle{empty}
\newtheorem{case}{Case}
\numberwithin{case}{proof}

% operators
\DeclareMathOperator*{\argmin}{arg\,min}
\DeclareMathOperator*{\idx}{idx}
\DeclareMathOperator*{\conv}{conv}

\loadglsentries{glossary}
\makeglossaries

\renewcommand{\epsilon}{\upvarepsilon}
%\newcommand{\overlap}{\topinset{\hspace{-0.4ex}\rule{0.7ex}{0.15ex}}{\rotatebox{60}{\rule{2ex}{0.15ex}}}{0.5ex}{0.6ex}}

\newcommand{\includeChapter}[1]{\glsresetall\include{#1}\cleardoublepage}

\definecolor{lightgray}{gray}{0.8}
\newcommand\VRule{\color{lightgray}\vrule width 0.5pt}

\begin{document}
\selectlanguage{ngerman}
\isodate
\pagenumbering{roman}
%\frontmatter

\setboolean{@twoside}{false}
\maketitle
\setboolean{@twoside}{true}

\thispagestyle{empty}
\vspace*{\fill}
\hspace*{\fill}
for my daughter---in the hope that she will understand sometime
\cleardoublepage

\subsubsection*{\centering Erklärung}
Ich versichere, die vorliegende Arbeit selbstständig und nur unter
Benutzung der angegebenen Hilfsmittel angefertigt zu haben. Bei den
Experimenten sind keine unbeteiligten Dreiecke zu Schaden gekommen.

\vspace{2em}

Winfried Hellmann\hfill\\
Braunschweig, den {\origdate\printdate{\docdate}}
\vspace{5em}
\cleardoublepage

% abstracts
\begin{minipage}{\textwidth}
	\selectlanguage{english}
	\begin{abstract}
	This thesis studies the \gls{MMLT} problem which has recently be proven
	to be NP-hard, which implies that there is no polynomial time algorithm
	for finding an optimal \gls{MMLT} solution unless P\(=\)NP. We verify
	that instances with randomly distributed points can be solved in polynomial
	time on average, however.
	\end{abstract}

  \glsreset{MMLT}
	\selectlanguage{ngerman}
	\begin{abstract}
	In dieser Arbeit beschäftigen wir uns mit dem \gls{MMLT} Problem, von
	dem kürzlich bewiesen wurde, dass es NP-schwer ist. Folglich gib es
	keinen Algorithmus, der immer eine optimale \gls{MMLT} Lösung in
	polynomieller Laufzeit findet, es sei denn P\(=\)NP. Wir überprüfen,
	dass Instanzen mit zufällig verteilten Punkten dennoch im Durchschnitt
	in polynomieller Zeit gelöst werden können.
	\end{abstract}
\end{minipage}
\thispagestyle{plain}
\cleardoublepage

\selectlanguage{english}
\tableofcontents
\cleardoublepage

\pagenumbering{arabic}
\cleardoublepage
\setcounter{page}{1}

%\mainmatter
\includeChapter{introduction}
\includeChapter{integer_programming}
\includeChapter{triangulations}
\includeChapter{mmlt}
\includeChapter{implementation}
\includeChapter{results}
\includeChapter{conclusion}

\begin{appendix}
  \chapter{Documentation}
\begin{verbatim}
  - generated with Doxygen
  - link to git 
  - see implementation chapter 
\end{verbatim}
\includepdf[pages={2-}]{doxygen.pdf}

  \chapter{Result Data}
\label{cha:result_data}

\begin{table}
  \todo[inline]{replace}
  \caption{\label{tab:segment_index}}
\end{table}

\begin{table}
  \todo[inline]{replace}
  \caption{\label{tab:segment_length}}
\end{table}

\end{appendix}

\addchap{Glossary}
\renewcommand*{\glossaryname}{Other Symbols}
\deftranslation{Glossary}{Other Symbols}
\printglossary[type=\acronymtype]
\printglossary[type=prop]
\printglossary[type=set]
\printglossary[type=main]

\printbibliography
%\cleardoublepage

%\listoftodos

\thispagestyle{empty}
\vspace*{\fill}
%\hspace*{\fill}\\
\noindent{\scriptsize
With the \gls{MMLT} algorithm at hand, Greg, who loves triangles \cite{greg}, finally
made his way to world domination.
}

\end{document}
