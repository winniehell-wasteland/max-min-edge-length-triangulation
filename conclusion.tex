\newgeometry{margin=3.5cm}
\chapter{Conclusion}
The claim of this thesis was that even though the \gls{MMLT} problem
was proven to be NP-hard, instances with randomly distributed points can
still be solved in polynomial time. Therefore we explored some properties
of the problem and consequently presented an algorithm. Its implementation
was run on uniformly distributed random points in the unit square, and
we compared its running time to solving the complete \gls{IP} representing
the \gls{MMLT} problem. The results show that indeed we can achieve a polynomial
running time on average, and that our approach is significantly better than
simply working with the \gls{IP}.

As a side effect of analyzing our algorithm, we found out that given randomly
distributed points \(P\), the shortest line segment \(s \in S\) of all line
segments \(S\) with endpoints in \(P\) that does not cross any longer line
segment \(\gls{scross} \in S\) (\(|s| < |\gls{scross}|\)) has a sub-linear
index in \(S\) sorted by length with respect to the number of points in \(P\).
We verified this assumption in our experiments but the proof is still open,
and we could not find any publications on that topic.

There are several motivations for future work on the \gls{MMLT} problem:
We have not shown that our algorithm is the best approach by giving a lower
bound for the running time on \gls{MMLT} instances with randomly distributed
points, such that there may be other algorithms with better asymptotic running
time. Additionally, our implementation can be improved by using heuristics
such as applying a greedy algorithm for Weighted Independent Set on the
Intersection Graph with the line segment lengths as weights. It is also
possible that the use of SAT solvers instead of \gls{IP} solvers leads to
better results as the SAT problem is more restricted.

Another way of improving our results would be to consider different
distributions for random point sets or other underlying geometries.
Common choices for random point sets are grids, clusters, and Gaussian
distribution. Instead of selecting all points from the unit square
one may as well consider rectangles, circles or geometric objects
of higher dimensions. Besides, the resulting data may benefit from using
accurate number types instead of double precision floats. In general,
it might also be of interest to run our algorithm on real world
instances---even though we could not find any applications for the the
\gls{MMLT} problem.

Furthermore, our algorithm may exploit certain structures (which we
do not know of yet) in the input or the Intersection Graph to avoid
solving the \gls{IP} at all. For instances with \(\idx(\gls{enose}) = 0\)
we do so already. Strategies like blue rule and red rule for
Minimum Spanning Trees may have analogies for the \gls{MMLT} problem.

Finally, some of our constructions and results can be applied to
Polygon Triangulations. The \gls{MMLT} problem is also NP-hard for
general polygons such that adapting our algorithm may lead to new
insights. In general only few things have to be changed to forbid
line segments crossing the boundary, and forcing the boundary to
be part of the solution.
