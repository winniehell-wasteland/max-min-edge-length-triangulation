\chapter{Introduction}
\ldots\todo[inline]{1 pages}

Triangulations, that is subdividing the plane (or a polygon) into
triangles, are a popular topic in computational geometry --- not
only because of their connections to other problems but also due to
their practical applications. They can be helpful as a preprocessing 
in other algorithms
or as a tool in geometric proofs. One popular example is the
artgallery problem \todo{cite}. Another area where triangulations
are widely used is mesh generation and approximation of complex
geometric structures. \todo{cite}

For different use cases the objectives for a triangulation vary. For
example, the widely known Delaunay triangulation
\cite[Section 9.2]{deberg_compgeom} tends to avoid `"skinny'"
triangles and is therefore useful for meshes. One can image many
different properties to be optimized: Edge lengths, triangle area,
angle, and degree in a vertex are some of them. Many have already
been looked into, but for some of them no application is known by
now --- so they remain theoretical problems. In
\cref{cha:triangulations} we will have a brief overview of different
kinds of triangulations.

This thesis will focus on the \gls{MMLT}. Stated an open problem in
1991 \cite{triangulation_minmax_length}, it has been proven to be
NP-complete in 2012 \cite{mmlt_complexity}. However our assumption
is that the hard instances are rare and that random instances can
be solved in polynomial time on average. Therefore we provide an
algorithmic idea in \cref{cha:mmlt} and its implementation in
\cref{cha:implementation}.

Even though there seems to be no known application of \gls{MMLT} yet,
it is Greg's \cite{greg} preferred kind of triangulation.

%---------------------------------------------------------------------##########
