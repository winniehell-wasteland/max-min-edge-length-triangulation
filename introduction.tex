\chapter{Introduction}
Triangulations, that is subdividing the plane (or a polygon) into
triangles, are a popular topic in computational geometry --- not
only because of their connections to other problems but also due to
their practical applications. They can be helpful as a preprocessing 
in other algorithms
or as a tool in geometric proofs. One popular example is the
artgallery problem \cite{artgallery}. Another area where triangulations
are widely used is mesh generation and approximation of complex
geometric structures. \cite{meshing}

For different use cases the objectives for a triangulation vary. For
example, the widely known Delaunay Triangulation
\cite[Section 9.2]{deberg_compgeom} tends to avoid ``skinny''
triangles and is therefore useful for meshes. One can imagine many
different properties to be optimized: Edge lengths, triangle area,
inner angles, and degree of a triangulation vertex are some of them. 
Many have already
been looked into, but for some of them no application is known by
now---so they remain theoretical problems. In \cref{cha:triangulations} 
we will have a brief overview of different
kinds of triangulations.

This thesis will focus on the \gls{MMLT} problem introduced in \cref{cha:mmlt}.
Stated an open problem in 1991~\cite{triangulation_minmax_length}, 
it has been proven to be NP-complete in 2012~\cite{mmlt_complexity}. 
This implies that the worst case running time for an algorithm solving
\gls{MMLT} is more than polynomial unless P\(\not=\)NP.
However our assumption is that the worst case instances (for example
those constructed in~\cite{mmlt_complexity}) are rare+
and that random instances can be solved in polynomial time on average. 

In \cref{cha:mmlt} we examine some properties of the \gls{MMLT}
problem and develop an algorithmic idea. The key observation, which we
were not able to prove, is that the shortest line segment having endpoints
in a given point set, and which is not crossed by a longer segment is likely
to be short. Our experimental results in \cref{cha:results} confirm this
assumption and lead to a sub-linear bound for the index of such a line segment
in a set sorted by length.

We implemented the algorithm presented in \cref{cha:mmlt} to compare
its running time to natural approaches such as formulating \gls{MMLT}
as a standardized optimization problem and passing it to a solver. The
components of our program are described in \cref{cha:implementation}.
For the geometric part of the problem, we made use of the well
established \gls{CGAL}~\cite{cgal}.

To ease approaching the combinatorial aspects of the \gls{MMLT} problem
and to agree on necessary definitions, we cover some of the basics in
\cref{cha:integer_programming}. This allows us to come up with a
solely combinatorial definition of Triangulation problems in
\cref{cha:triangulations}, which we will make extensive use of later
within this thesis. In fact, abstracting geometric aspects of
Triangulation problems grants more flexibility, e.g. when transferring
approaches from the plane to higher dimensions. However, this
generalization comes to the prize of higher computational complexity.

% Even though there seems to be no known application of \gls{MMLT} yet,
% it is Greg's \cite{greg} preferred kind of triangulation.

%---------------------------------------------------------------------##########
