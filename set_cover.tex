\chapter{Set Cover}
In the next recess Greg and his friends were discussing again what
to play. It was always hard to find games that everybody liked.
For example, Greg likes to play tables tennis or Duck, Duck, Goose but
does not want to play soccer. Suddenly Greg had an idea: What about
making a list of everybody's first and second choice and then 
finding groups which could play together. His friends agree and
you can see the result in \cref{tab:game_voting}.

\begin{table}[ht]
  \centering
  \begin{tabular}{l|cccc}
    & table tennis & Duck, Duck, Goose & soccer & swings \\
    \hline
    Greg  & 1 &   &   & 2 \\
    Joey  & 2 & 1 &   &   \\
    Susan & 1 &   & 2 &   \\
    Bella &   &   & 2 & 1 \\
    Jenny & 1 & 2 &   &   \\
    Jason &   &   & 1 & 2 \\
    Alice &   & 2 &   & 1 \\
    Bob   &   & 1 & 2 &   \\
  \end{tabular}
  \caption{\label{tab:game_voting}Voting of Greg and his friends}
\end{table}

Then Greg went to Mrs. Lloyd and asked her how they could find the 
best solution. Firstly, she made another list
(\cref{tab:example_set_cover}) where she put all the children's
names for each game and summed up their preferences.%
\footnote{Note that I cheated here: Usually you would divide the
sum of preferences by the number of children who want to play the
game. But Greg did not know yet how to divide, so I made all the
groups have equal size.}

\begin{table}[ht]
  \centering
  \begin{tabular}{lc}
    table tennis (Greg, Joey, Susan, Jenny): & 4 \\
    Duck, Duck, Goose (Joey, Jenny, Alice, Bob): & 6 \\
    soccer (Susan, Bella, Jason, Bob):       & 7 \\
    swings (Greg, Bella, Jason, Alice):      & 6 \\
  \end{tabular}
  \caption{\label{tab:example_set_cover}Summed up preferences for each game}
\end{table}

Mrs. Lloyd showed the children that there were no two games to cover
all of them --- so they needed to split up into at least three groups.
The best way to do so was to play table tennis, Duck, Duck, Goose, and
on the swings. That was everybody's first choice besides for Jason.
The problem Mrs. Lloyd solved is an instance of the minimum weight 
set cover problem.

\begin{problem}[Minimum Weight Set Cover]\label{prob:mwsc}
  \hfill
  \begin{labeling}{\hspace{4em}}
    \item[\textbf{Given:}]
      A ``universe'' (set) of objects \(U\), 
      subsets \(S = \{S_i\}\)
      such that \(\bigcup\limits_{S_i \in S} S_i = U\),
      and a weight function \(c : S \to \gls{R}_+\)
    \item[\textbf{Sought:}]
      A set \(R \subseteq S\)
      which covers the universe, i.e. 
      \(\bigcup\limits_{S_i \in R} S_i = U\)
      and minimizes \(\sum\limits_{S_i \in R} c(S_i)\)
  \end{labeling}
\end{problem}

\Cref{prob:mwsc} is NP-hard \cite{set_cover} and the related decision 
problem was already one of the problems Karp has shown to be 
NP-complete \cite{set_cover_decision}.

%---------------------------------------------------------------------##########
